\documentclass[12pt, letterpaper, twoside]{article}
\usepackage[utf8]{inputenc}
\usepackage{hyphenat}
\usepackage[english]{babel}
\useshorthands{~}
\defineshorthand{~-}{\hyp{}}

\title{Small-scale Numerical Weather Forecast Model}
\author{Yan Liu \thanks{email:yliu.yb@qq.com}}
\date{\today}

\begin{document}
\maketitle
\newpage
\section*{config module}
\&time control
\newline 
start date time, 
end date time, 
run hours, 
time interval(unit:seconds)
\newline
\&grid control
\newline
center\_point\_lon, 
center\_point\_lat,
center\_bottom\_height(unit:m), 
x-direction grid num,
y-direction grid num,
z-direction grid num,
x, y, z interval distance(unit:m)

\section*{input data module}
\&common data
\newline
surface pressure,
height
\newline
\&initial data variable 
\newline 
density, 
temperature, 
x-component velocity,
y-component velocity, 
vertical velocity,
specific humidity. (3 dimension)
\newline
\&force data variable
\newline 
density, 
temperature, 
x-component velocity,
y-component velocity, 
vertical velocity,
and specific humidity 
at 
west, 
east, 
south, 
north, 
bottom and top boundary, separately.
\section*{dynamic solver module}
r\_old = r\_initial
\newline
time loop:
\newline
begin t
\newline
\hspace*{5mm}explicit R-K loop:
\newline
\hspace*{5mm}begin R-K
\newline
\hspace*{10mm}r*=r\_old-FVM(r\_old, boundary\_t) for R-K1
\newline
\hspace*{10mm}r**=r\_old-FVM(r*, boundary\_t) for R-K2
\newline
\hspace*{10mm}r\_new=r\_old-FVM(r**, boundary\_t) for R-K3
\newline
\hspace*{5mm}end R-k
\newline
\hspace*{5mm}output(r\_new)
\newline
\hspace*{5mm}r\_old = r\_new
\newline
end t
\section*{Fortran module}
customType.f90

\end{document}